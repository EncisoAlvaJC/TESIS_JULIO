%%%%%%%%%%%%%%%%%%%%%%%%%%%%%%%%%%%%%%%%%%%%%%%%%%%%%%%%%%%%%%%%%%%%%%%%%%%%%%%%%%%%%%%%%%%%%%%%%%%
%%%%%%%%%%%%%%%%%%%%%%%%%%%%%%%%%%%%%%%%%%%%%%%%%%%%%%%%%%%%%%%%%%%%%%%%%%%%%%%%%%%%%%%%%%%%%%%%%%%

\chapter{Introducción}

Gracias a los avances médicos del último siglo se ha incrementado la esperanza de vida y la 
calidad de vida; desafortunadamente también ha aumentado la presencia de enfermedades 
no-transmisibles asociadas con la edad. 
%
Para muchas de esas enfermedades no se han identificado factores causales o curas definitivas 
\cite{PlanAlzheimer04}.
%
En México el sector de la población con más de 60 años de edad (aquellos con alto riesgo para este
tipo de enfermedades) contempló a 10 millones de personas en 2010; se estima en 2015 esta cifra 
creció a 12 millones \cite{Censo10,Intercensal15}.

De entre las enfermedades ante las cuales este grupo de edad es vulnerable, en este trabajo se 
destaca la demencia. 
%
La demencia consiste en el desarrollo de déficit cognoscitivos suficientemente graves como para 
interferir en las actividades laborales y sociales.
%
El deterioro cognitivo característico de la demencia se considera irreversible, debido a lo cual 
ha surgido un gran interés en definir y diagnosticar etapas tempranas de este padecimiento con el 
fin de evitar en lo posible dicho síntoma \cite{Knopman01}.

Se define entonces al deterioro cognitivo leve (DCL) como \textit{una alteración adquirida y 
prolongada de una o varias funciones cognitivas, que no corresponde a un síndrome focal y no cumple 
criterios suficientes de gravedad para ser calificada como demencia} \cite{Robles02}.
%
No hay un consenso absoluto sobre qué deficiencias cognitivas --o más bien en qué grado-- 
distinguen a un individuo con DCL, de modo que hay una multitud de definiciones que no son
equivalentes \cite{Petersen01}.
%
En el presente trabajo se favorecen los métodos para caracterizar --y diagnosticar-- el DCL en base 
a mediciones objetivas, pero manteniendo presente que el fenómeno del deterioro cognitivo no puede 
reducirse exclusivamente a tales mediciones; las conclusiones sobre las señales electrofisiológicas 
deben ser contrastadas, por ejemplo, con los resultados de evaluaciones neuropsicológicas.

En concreto se utilizarán registros de electroencefalograma (EEG) obtenidos durante etapas
específicas en el sueño del paciente, técnica conocida como polisomnografía (PSG).
%
Conviene destacar que muchos de los marcadores para el DCL definidos en base al EEG, dependen
efectivamente de su respectivo espectro de potencias; el razonamiento usual para ello es que si el 
EEG está asociado al grado de actividad cerebral en términos de energía, entonces el espectro de 
potencias explica \textit{cómo} es dicha actividad.

El presente trabajo toma parte en el problema metodológico de que las señales 
electrofisiológicas típicamente representan procesos no--lineales y no--estacionarios, y sin 
embargo suelen ser analizadas usando herramientas que suponen linealidad y estacionariedad.
%
En el caso particular del espectro de potencias, es común que sea calculado usando la 
transformada de Fourier sobre segmentos cortos para evitar los \textit{efectos} de la 
no--estacionariedad \cite{Kaiser00}.
%
A consecuencia de lo anterior los datos pueden contener información \textit{oculta}, o incluso 
pueden llegar a no ser representativos del fenómeno que se estudia. 
%
%Es por ello que se buscan herramientas para verificar la estacionariedad débil en los registros
%electrofisiológicas, y con especial atención en la posibilidad de que puedan usarse como marcadores
%de deterioro cognitivo.
%%
%Adicionalmente, la posibilidad de que sujetos con PDC exhiben estacionariedad débil en sus 
%registros de EEG en mayor proporción (respecto a individuos sanos) fue sugerida anteriormente
%\cite{Cohen77}.

%%%%%%%%%%%%%%%%%%%%%%%%%%%%%%%%%%%%%%%%%%%%%%%%%%%%%%%%%%%%%%%%%%%%%%%%%%%%%%%%%%%%%%%%%%%%%%%%%%%
%%%%%%%%%%%%%%%%%%%%%%%%%%%%%%%%%%%%%%%%%%%%%%%%%%%%%%%%%%%%%%%%%%%%%%%%%%%%%%%%%%%%%%%%%%%%%%%%%%%

\section{Antecedentes}

El sueño MOR ha sido ampliamente reconocido como parte de la consolidación de la memoria, así como
otras funciones cognitivas 
\cite{Fishbein1971,Fishbein1977,Lucero1970,Pearlman1971,Pearlman1974,Smith1991}.
%
En el caso de adultos mayores, la correlación entre deterioro cognitivo y trastornos del sueño ha 
sido reportada por varios autores a partir de estudios poblacionales 
\cite{Amer13,Miyata13,Reid06,Potvin12}.
%
Tal correlación era de esperarse ya que los procesos de atención y memoria, por ejemplo, dependen de 
los circuitos colinérgicos activados durante el sueño MOR \cite{Braun1997}; estos circuitos son 
propensos a degradación estructural tanto en el envejecimiento normal como en el patológico,  y 
especialmente en el segundo \cite{Schliebs11}.

En 2016 Vázquez-Tagle y colaboradores estudiaron la epidemiología del DCL en adultos mayores dentro 
del estado de Hidalgo y su posible relación con trastornos de sueño, encontrando efectivamente una 
correlación entre una menor eficiencia del sueño (porcentaje de tiempo de sueño respecto al tiempo 
en cama) y la presencia de deterioro cognitivo \cite{VazquezTagle16}.
%
En aquél estudio se efectuaron registros de PSG para algunos de los participantes, con la intención 
de verificar que existen diferencias en los registros correspondientes a individuos con y sin DCL.
%
El presente trabajo se enmarca dentro de una colaboración con los responsables del estudio 
mencionado, con el objetivo de identificar concretamente los posibles cambios en los registros de PSG 
ocurridos durante el DCL.

La idea de que sujetos con deterioro cognitivo exhiban cambios en sus registros de PSG relacionados 
a la estacionariedad débil, fue sugerida por Cohen en 1977 \cite{Cohen77}; aquél análisis se 
refiere a su vez a trabajos anteriores sobre estacionariedad y normalidad en registros de EEG
\cite{McEwen75,Sugimoto78,Kawabata73}.
%
Los estudios referidos se enmarcan en un primer intento de verificar que los registros 
electrofisiológicos no pueden modelarse como señales \textit{simples} (lo contrario a señales 
complejas).
%
El tema de la estacionariedad en señales parece relevante nuevamente a la luz de una revisión
por Kreuz y colaboradores \cite{Kreuz07}, según la cual el método más eficaz para detectar 
correlación es la correlación \textit{clásica} --en comparación con algunos métodos basados en 
entropía o información mutua, por ejemplo. 
%
Para llegar a tal conclusión analizaron algunos tipos \textit{comunes} de datos, experimentales y 
simulados, así como algunos tipos comunes de perturbaciones.

Un resultado tan controversial provoca replantearse, por ejemplo, si algún método en particular es 
adecuado para un tipo arbitrario de datos, o si algunas generalizaciones para estos métodos son 
efectivamente necesarias.
%
Claramente una grupo tan vasto de interrogantes es efectivamente inútil como motivo de
investigación; se presenta como tal con base a los resultados obtenidos.

%%%%%%%%%%%%%%%%%%%%%%%%%%%%%%%%%%%%%%%%%%%%%%%%%%%%%%%%%%%%%%%%%%%%%%%%%%%%%%%%%%%%%%%%%%%%%%%%%%%
%%%%%%%%%%%%%%%%%%%%%%%%%%%%%%%%%%%%%%%%%%%%%%%%%%%%%%%%%%%%%%%%%%%%%%%%%%%%%%%%%%%%%%%%%%%%%%%%%%%

\section{Pregunta de investigación}

¿Los registros de polisomnograma en adultos mayores pueden considerarse como series de tiempo 
débilmente estacionarias?
%
¿Es posible que tal caracterización se vea influida por el estado cognitivo del sujeto?

%%%%%%%%%%%%%%%%%%%%%%%%%%%%%%%%%%%%%%%%%%%%%%%%%%%%%%%%%%%%%%%%%%%%%%%%%%%%%%%%%%%%%%%%%%%%%%%%%%%
%%%%%%%%%%%%%%%%%%%%%%%%%%%%%%%%%%%%%%%%%%%%%%%%%%%%%%%%%%%%%%%%%%%%%%%%%%%%%%%%%%%%%%%%%%%%%%%%%%%

\subsection{Hipótesis}

Existen diferencias en la actividad eléctrica cerebral en adultos mayores con PDC, respecto a 
individuos sanos, y es posible detectar dichas diferencias como una mayor o menor 
\textit{presencia} de estacionariedad débil en registros de PSG durante el sueño profundo.

%%%%%%%%%%%%%%%%%%%%%%%%%%%%%%%%%%%%%%%%%%%%%%%%%%%%%%%%%%%%%%%%%%%%%%%%%%%%%%%%%%%%%%%%%%%%%%%%%%%

\subsection{Objetivo general}

Obtener pruebas estadísticas formales para detectar si una serie de tiempo dada procede de un
proceso estocástico débilmente estacionario.
%
Usar tales pruebas sobre registros de polisomnograma en adultos mayores, para investigar si la 
presencia de segmentos débilmente estacionarios se correlaciona con la condición de probable
deterioro cognitivo.

%%%%%%%%%%%%%%%%%%%%%%%%%%%%%%%%%%%%%%%%%%%%%%%%%%%%%%%%%%%%%%%%%%%%%%%%%%%%%%%%%%%%%%%%%%%%%%%%%%%

\subsection{Objetivos específicos}

\begin{itemize}
\item Estudiar la definición de estacionariedad para procesos estocásticos.

\item Investigar cómo detectar, como prueba de hipótesis, si una serie de tiempo dada proviene
de un proceso estocástico débilmente estacionario, y bajo qué supuestos 
es válida dicha caracterización.

\item Decidir si los registros de PSG, durante sueño profundo, son débilmente estacionarios.

\item Investigar si la presencia de segmentos estacionarios en los registros es diferente si el
PSG corresponde a un individuo con PDC.
\end{itemize}

%%%%%%%%%%%%%%%%%%%%%%%%%%%%%%%%%%%%%%%%%%%%%%%%%%%%%%%%%%%%%%%%%%%%%%%%%%%%%%%%%%%%%%%%%%%%%%%%%%%
%%%%%%%%%%%%%%%%%%%%%%%%%%%%%%%%%%%%%%%%%%%%%%%%%%%%%%%%%%%%%%%%%%%%%%%%%%%%%%%%%%%%%%%%%%%%%%%%%%%